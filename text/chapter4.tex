\section{Diskussion der Ergebnisse}
\subsection{Bewertung des Prototyps}
Zur methodischen Erstellung eines Prototyps gehört auch, diesen strukturiert zu bewerten. Ziel der Methode war es, Erkenntnis über den technischen Prozess der Implementierung einer \ac{EDA} zu erlangen und zu zeigen, dass dies im gegebenen Softwareumfeld überhaupt möglich ist. 
Gezeigt wurde zuerst einmal, dass es möglich ist, eine \ac{EDA} zu implementieren. Die Erkenntnis bezüglich des technischen Prozesses ist unter Abschnitt \ref{implementierung} dokumentiert. Hierzu muss aber erwähnt werden, dass die beschriebenen Schritte nur für ein relativ spezifisches System Gültigkeit haben. Voraussetzung ist, dass ein SAP On-Premise System sowohl als Producer als auch als Subscriber dienen soll und als Event-Platform der BTP-Servie Event Mesh verwendet werden soll. Das Beispiel deckt beispielsweise weder die Möglichkeit ab, externe Producer zu registrieren, noch ein SAP S/4 Cloud System als Subscriber zu verwenden. 
Trotzdem kann die gewonnene Erkenntnis auch in anderen Szenarien teilweise wiederverwendet werden, da die verwendeten Softwarekomponenten im Sinne der \ac{EDA} prinzipbedingt modular verwendet werden können und somit auch die Implementierung zu in anderen Szenarien in beschriebener Form wiederkehrt.
Der praktische Nutzen des Prototyps besteht also in seiner Funktion als Systemdemo von \ac{EDA} für das System der in der Einleitung beschriebenen SAP HCM Anwendung, könnte aber auch darüber hinaus noch Anwendung finden.

\subsection{Beurteilung von EDA und REST im Kontext des Anwendungsbeispiels}
Die Fragestellung der Arbeit war, ob es sich lohnt, die \ac{EDA} in komplexen Softwareumfeldern einzusetzen. In den theoretischen Grundlagen der Arbeit wurden einige Vorteile dieses Architekturansatzes dargestellt, die im Prototypen überprüft werden können. Konkret sollen folgende Vorteile überprüft werden:
\begin{itemize}
    \item Modularisierung
    \item Ausfallsicherheit
    \item Integrationsmöglichkeiten
\end{itemize}
\subsubsection*{Modularisierung}
Die Modularisierung ist ein zentraler Vorteil der \ac{EDA}. Durch die lose Kopplung der Komponenten und die Verwendung von Events als Schnittstelle wird die Modularisierung der Softwarearchitektur gefördert. Dieser Vorteil tritt im Prototypen in der Entkopplung der Producer und Subscriber Applikationen zutage. Hierbei wurden das \ac{BO}, welches das Ereignis produziert vollständig unabhängig von der Subscriber Applikation entwickelt. Es gibt keine Überschnitte bezüglich der Code-Basis, aber auch zeitlich wurden die beiden Komponenten unabhängig voneinander entwickelt. Es ist also durchaus auch denkbar, dass wenn verschiedenen Teams an der Entwicklung der Anwendungen beteiligt sind, diese ohne größere Abstimmung, also unabhängig entwickeln könnten. Das ist ein großer Vorteil, da es die Entwicklungsgeschwindigkeit erhöht und die Komplexität der Software reduziert.
\subsubsection*{Ausfallsicherheit}
Aus der Modularisierung folt logisch auch eine höhere Ausfallsicherheit der Software. In der Entwicklung trat das hervor, als beide Softwarkomponenten vollständig unabhängig voneinander getestet werden konnten, ohne dass sie aufeinander Einfluss nehmen mussten. Durch die Möglichkeit, in der Event-Platform Ereignisse zu Testzwecken auszulösen, konnten das Verhalten der Applikationen getestet werden ohne das eine Abhängigkeit zu anderen Komponenten bestand. Dieses Beispiel zeigt, dass, selbst wenn eine Softwarekomponente nicht läuft, die andere davon nicht beeinträchtigt wird. Im Beispiel könnte also der vom \ac{BO} bereitgestellte Service weiterlaufen, auch wenn die vom Subscriber bereitgestellte Benachrichtigung gesendet wird. Es könnten also weiterhin Travel-Objekte angelegt, geändert und gespeichert werden, es würden nur keine Benachrichtigungen mehr gesendet. Das erhöht die Verfügbarkeit des Dienstes, weil nicht jede einzelne Funktionalität zum Totalausfall des Systems führen kann.
\subsubsection*{Integrationsmöglichkeiten}
Die Integrationsmöglichkeiten sind ein weiterer Vorteil der \ac{EDA}. In der Implementierung des Prototyps ist die Sprache auf den Cloud-Events Standard gekommen, der in der Industrie übergreifend verwendet wird. Nativ bietet das SAP-Event Mesh die Möglichkeit über eine API oder sogenannte Webhooks auch externe Komponenten anzubinden. In der Entwicklung des Prototyps wurde diese Möglichkeit nicht weiter getestet, die Recherche hat aber immer wieder gezeigt, dass die Integrationsmöglichkeiten des Event-Mesh sehr umfangreich sind. Eine Möglichkeit zur Weiterentwicklung des Prototyps ist es also, auch externe Publisher und Subscriber einzubinden.

\subsection{Chancen der Technologie im betriebswirtschaftlichen Kontext}
\subsubsection*{Methodik}
Auf Grundlage eines Experteninterviews zu dem Thema sollen die bisherigen Erkenntnisse auf einen umfassenderen Rahmen übertragen werden und aus betriebswirtschaftlicher Sicht bewertet werden. 
\subsubsection*{Schwierigkeiten bei komplexen Softwareumfeldern}
\subsubsection*{Vorteile der EDA}
