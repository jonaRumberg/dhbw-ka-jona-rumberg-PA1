\section{Schlussbetrachtung}
\subsection{Zusammenfassung der wichtigsten Ergebnisse}
\subsection{Abschließende Beurteilung von EdA. und REST}
\subsubsection{Praktischer Nutzen}
\subsubsection{Nachteile und Fallstricke in der Entwicklung}
\subsubsection{Weiterführende Chancen}
\subsection{Kritische Reflexion der Arbeit}
