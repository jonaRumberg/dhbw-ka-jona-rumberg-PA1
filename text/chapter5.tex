\section{Resümee}
\subsection{Zusammenfassung der wichtigsten Ergebnisse}
Im praktischen Teil der Arbeit wurden die theoretischen Ergebnisse im Zuge eines Prototyps angewendet und als 'Proof of Concept' realisiert. Die Implementierung bezieht sich auf den Kontext einer SAP-OnPremise Anwendung je als Producer und Subscriber und dem SAP BTP Event-Mesh als Event Platform. Es wurde evaluiert, welche Artefakte erstellt und Konfigurationsschritte vorgenommen werden müssen, um eine \ac{EDA} in diesem Kontext zu implementieren. Als praktische Erkenntnis konnte aus diesem Prozess gezogen werden, dass es, in dem gegeben Rahmen, möglich ist, eine \ac{EDA} zu implementieren. \\
Weiterhin konnten anhand des Prototyps Eigenschaften einer \ac{EDA} beobachtet werden, die vorher in der Theorie besprochen wurden. Hierbei wurden die Modularisierung, die Ausfallsicherheit und die Integrationsmöglichkeiten der \ac{EDA} näher betrachtet und im Kontext des Prototyps eingeordnet. Sowohl Modularisierung und Ausfallsicherheit konnten klar beim Prototypen beobachtet werden, Integrationsmöglichkeiten konnten im Ausblick auf die Weiterentwicklung des Prototyps erkannt werden. \\
Die bis dorthin gesammelten Erkenntnisse wurde abschließend in einen betriebswirtschaftlichen Kontext eingeordnet. Als zentrale Faktoren wurden hier die reduzierte Softwarekomplexität und die gesteigerte Agilität des Geschäfts ausgemacht. Außerdem wurde ein Blick auf eine aktuelle Studie zur Verbreitung von \ac{EDA} in der Industrie geworfen, die zeigt, dass die Technologie in der Industrie bereits weit verbreitet ist.

\subsection{Handlungsempfehlung}
Die bisherige Arbeit lief zielstrebig auf eine uneingeschränkte Empfehlung für Architekturfragen hinaus, es soll aber auch ein kritischer Blick auf die Technologie geworfen werden. 
Beleuchtet wurden vor allem die Vorteile von \ac{EDA} in komplexen Softwareumfeldern. Da der Ansatz bedingt, dass im kompletten Entwurfsprozess auf dessen Prinzipien geachtet werden muss, ist er für kleiner Softwarevorhaben meist nicht sinnvoll. Die besprochenen Vorteile wirken sich am stärksten in Systemen mit vielen Schnittstellen und Komponenten aus. In einem weniger komplexen System würde alleine die Einführung einer Event-Platform als zusätzliche Komponente die Komplexität unnötig aufblähen. Würde in der Konzeption von kleinen Softwarelösungen konsequent ereignisgesteuert gearbeitet, könnte das sogar dazu führen, dass eigentlich simple Prozesse in zu viele kleine technische Ereignisse aufgebrochen werden und die Übersichtlichkeit vollständig verloren geht. 
Zusätzlich muss erwähnt werden, dass der Ansatz am sinnvollsten im Umfeld verteilter Systeme ist. In einer monolithischen Anwendung, die nur aus einer Komponente besteht, verkompliziert eine \ac{EDA} die Architektur unnötig. 
Als abschließende Einschränkung muss erwähnt werden, dass eine Migration eines bestehenden Systems auf eine \ac{EDA} nur bedingt sinnvoll ist. Da die \ac{EDA} ein auf sehr grundlegender Ebene unterschiedlicher Ansatz zu anderen Architekturansätzen ist, würde eine Migration eines bestehenden Systems einen meist erheblichen Aufwand nach sich ziehen. Vor allem wenn viele unterschiedliche Schnittstellen betroffen wären, ist es sehr aufwändig all diese Schnittstellen auf die neuen Prinzipien umzustellen. 
Insgesamt ist die \ac{EDA} ein Architekturansatz, der in komplexen Softwareumfeldern, die aus vielen Komponenten bestehen und viele Schnittstellen haben, sinnvoll eingesetzt werden kann. Speziell bei der Neukonzeption von Systemen ist es deshalb sinnvoll, die \ac{EDA} als Architekturansatz in Betracht zu ziehen. Migrationen auf eine \ac{EDA} sind meist nur dann sinnvoll, wenn das bestehende System bereits Merkmale einer \ac{EDA} aufweist.

\subsection{Kritische Reflexion der Arbeit und Ausblick}
Die Arbeit hat sich mit der Frage beschäftigt, ob es sinnvoll ist, die \ac{EDA} in komplexen Softwareumfeldern einzusetzen. Die Fragestellung wurde in der Arbeit beantwortet, es wurde gezeigt, dass die \ac{EDA} in komplexen Softwareumfeldern sinnvoll eingesetzt werden kann. Der erstellte Prototyp wurde dazu verwendet ein näheres technisches Verständnis der theoretischen Konzepte zu erreichen. Die Erkenntnisse aus der Implementierung wurden in der Arbeit dokumentiert und in einen betriebswirtschaftlichen Kontext eingeordnet. \\
Möglichkeiten zur weiteren Forschung gibt es, wie schon in der Evaluation des Prototyps erwähnt, auf technischer Seite beispielsweise bei der Erforschung weiterer Integrationsmöglichkeiten. Zudem könnten noch andere Systemlandschaften betrachtet werden, um die Ergebnisse der Arbeit zu verifizieren. \\
Auf betriebswirtschaftlicher Seite könnte die Arbeit um eine Kosten-Nutzen-Analyse erweitert werden. Hierbei könnte beispielsweise untersucht werden, ob die Vorteile der \ac{EDA} die Kosten für die Einführung der Technologie übersteigen. \\
