\section{Theoretischer Hintergrund}
\subsection{\ac{EDA}}
\subsubsection*{Fragen an diesen Abschnitt}
\begin{itemize}
    \item Was ist \ac{EDA}?
    \item Was ist ein Event?
    \item Warum ist \ac{EDA} relevant?
\end{itemize}
\subsubsection*{Ereignisorientierung als Architekturansatz}
Zuerst einmal handelt es sich bei \ac{EDA} um ein Konzept der Prozessmodellierung. Im Gegensatz zur gewöhnlichen Ablauf-orientierten Modellierung werden die Prozesse nicht als aufeinanderfolgende Schritte, sondern als Reaktionen auf Zustände konzeptioniert. Daraus resultiert, dass nicht mehr die prozedurale Abhandlung von Arbeitsschritten die zentrale Aufgabe in der Anwendungssystem-Entwicklung darstellt, sondern die Reaktion auf Ereignisse. Im Mittelpunkt von Architekturentscheidungen steht die Frage: "Was passiert, wenn dieses Ereignis eintritt?" und nicht mehr: "Welche Schritte müssen zur Erfüllung dieser Anforderung gegangen werden?". Was daraus resultiert, ist eine Architektur, die schon mit Beginn der Konzeption wesentlich agiler und robuster ist, da von Anfang an mit der Annahme gearbeitet wird, dass prinzipiell zu jedem Zeitpunkt jedes Ereignis eintreten kann.\footcite[Vgl.][S.30]{EDA2010}
\subsubsection*{title}
\subsection{RESTful \ac{API}}
\subsection{Technologie im Anwendungsbeispiel}
\subsection{Forschungsmethodik}
\subsection{Zusammenfassung des theoretischen Teils}



 
