\section{Einleitung und Grundlagen der Betrachtung}

\subsection{Unternehmensprofil}
Die SAP SE ist ein deutsches Softwareunternehmen, das seit 1972 Unternehmenssoftware entwickelt. Heute beschäftigt es rund 105000 Mitarbeiter und hat Standorte weltweit. Die \ac{ERP} Systeme der Firma haben in der Geschäftswelt entscheidenden, branchenübergreifenden Einfluss. SAP bietet hierbei Möglichkeit, durch umfassende Funktionen und eine einheitliche, integrierte Datenbasis, Geschäftsprozesse zu überblicken, dieses digital abzuwickeln und zu automatisieren. \footcite[Vgl.][]{sapse_was}

% \epigraph{"`Des Menschen größtes Verdienst bleibt wohl, wenn er die Umstände soviel als möglich bestimmt und sich so wenig als möglich von ihnen bestimmen läßt."'}{Johann Wolfgang von Goethe\footcite[S. 10]{Freund2014}}
% Ein zum Thema passendes Zitat fast immer eine gute Einleitung für die Arbeit. 
% \ac{BPMN} ist eine Modellierungssprache.\footcite[Vgl.][S. 1]{Freund2014} Bei der ersten Verwendung von Abkürzungen werden diese in Klammern automatisch ausgeschrieben. 
% Bei der zweiten Verwendung ist das nicht so, wie man anhand von \ac{BPMN} sehen kann. "`Das ist ein direktes Zitat aus dem Internet"'.\footcite[S. 3]{OMG2018}
% Bei einseitigen Quellen kann man die Seitenzahl weglassen.\footcite[Vgl.][]{schlechteQuelle}
% Es gibt viele schlechte Quellen.\footcite[Vgl.][S. 1-3]{schlechteQuelle2}

\subsection{Motivation und Problemstellung}
Die wohl umfassendste Entwicklung in der globalen \ac{IT}-Branche ist heutzutage wohl die in Richtung Cloud. Um Geschäftsprozesse in diesem Umfeld umfassend abzubilden ist es immer wieder nötig verschiedene Dienste miteinander zu integrieren und so ist die Entwicklung sogenannter APIs zu einem Kernbestandteil der modernen Entwicklungstätigkeit geworden.
\\Ein weit verbreitetes Design Pattern ist dabei die sogenannte \ac{REST} \ac{API}. In der Migration von ehemals sequenziell ablaufenden Prozessen kann es jedoch zu Hürden kommen, da eine \ac{REST} \ac{API} von Grund auf stateless angelegt sein sollte. Hier kommt Event basiertes Design ins Spiel, das dieses Problem lösen könnte.

% Abbildungen und Tabellen sind natürlich auch möglich.

% \begin{figure}[H]
%   \centering
% 	\includegraphics[width=0.4\textwidth]{DHBW_logo.pdf}
%    \caption[Das Logo der DHBW]{Das Logo der DHBW\footnotemark}
% \end{figure}
% \footnotetext{\cite[][S.1]{DHBW}}

% Als Grafikformate werden u.a. PDF, PNG und JPEG akzeptiert. Die Bilddatei muss im Order "`images"' liegen.
% Mit einem Label in einer Abbildung oder Tabelle kann man darauf referenzieren, wie man an der Abbildung \ref{AbbildungLogoMusterfirma} sehen kann.
% \begin{figure}[H]
%   \centering
% 	\includegraphics[width=0.3\textwidth]{company_logo.pdf}
%    \caption[Das Logo der Musterfirma]{Das Logo der Musterfirma\protect\footnotemark}
%    \label{AbbildungLogoMusterfirma}
% \end{figure}
% \footnotetext{Eigene Darstellung in Anlehnung an \cite[S.4]{Freund2014}}

% Die Breite einer Grafik oder einer Tabelle lässt sich einfach als Faktor festlegen. 1 entspricht dabei der Textbreite und 0.5 die Hälfte der Textbreite.
% Bei Tabellen wird die angegebene Breite nur bei Bedarf ausgenutzt.

% \begin{table}[H]
%   \centering
%   \begin{tabulary}{0.7\textwidth}{|L|L|}
%   \hline 
%   \rowcolor{tableHeading}Eigenschaft & Wert \\ 
%   \hline 
%   Größe & 20 cm \\ 
%   \hline 
%   Gewicht & 1 kg \\
%   \hline
%   Haarfarbe & braun \\  
%   \hline 
%   \end{tabulary} 
%   \caption[Eine Tabelle ohne Quellenangabe]{Eine Tabelle ohne Quellenangabe}
% \end{table}


% Experteninterviews.\footcite[Vgl.][S. 7]{Meuser2009} Ein Zitat aus der Wirtschaftswoche.\footcite[Vgl.][S. 32]{WIWO2018}
% Firmeninternes Material kann auch zitiert werden.\footcite[Vgl.][]{Firma2018} "`Das Zitat stammt aus einem Interviewprotokoll"'.\footcite{Experte2018}

% Mit zwei Backslash \\ erzwingt man einen Zeilenumbruch. Bei langen Wörtern funktioniert die Worttrennung oftmals nicht mehr.
% Dann muss man selbst die Silbentrennung vornehmen: Donau\-dampf\-schiff\-fahrts\-gesell\-schafts\-kapitän. \\
% Aufzählungen:
% \begin{itemize}
%   \item Punkt 1
%   \item Punkt 2
% \end{itemize}
% Nummerierte Aufzählung:
% \begin{enumerate}
%   \item Punkt 1
%   \item Punkt 2
% \end{enumerate}

% Fußnoten sind besonders praktisch für Verweise auf andere Abschnitte der Arbeit.\footnote{Siehe Abschnitt \ref{Abschnitt:Arbeitsumfeld}} 
% Mit dem ref-Befehl lassen sich Labels referenzieren. Das funktioniert bei Abbildungen, Tabellen, Kapiteln und Abschnitten.

\subsection{Zielsetung}
Das Ziel dieser Arbeit ist die Untersuchung vom Zusammenspiel des eventbasierten Ansatzes und dem \ac{REST}ful design pattern, wobei speziellen Wert auf die Betrachtung der Robustheit und Fehlerresillienz des entstehenden Systems gelegt werden soll. Zudem soll ein solches System im Kontext einer HR-Anwendung beispielhaft umgesetzt werden.
\subsection{Abgrenzung}
\subsection{Vorgehensweise}