\section{Einleitung}


\subsection{Motivation und Problemstellung}
Eine der größten Schwierikeiten bei der Entwicklung von Softwareprojekten in größerem Maßstab ist es, die Komplexität der Software zu beherrschen. Schnell passiert es, dass über die Verbindung von vielen verschiedenen Softwarekomponenten über verschiedenen Schnittstellen die Komplexität der Software exponentiell ansteigt. Dies führt zu einer Reihe von Problemen, die die Entwicklung und den Betrieb der Software erschweren. \citepls Einen Ansatz zur Beherrschung dieser Komplexität bietet die \ac{EDA}. Sie verspricht, durch den Fokus auf Ereignisse bei der Systemarchitektur eine Reihe von Vorteilen. In der Prozessmodellierung lassen sich Geschäftsvorfälle einfacher modellieren, in der Implementierung wird von Beginn an eine modulare Struktur geschaffen, die Ausfallsicherheit, Integrationsmöglichkeiten und eine bessere Lesbarkeit des Programmcodes bietet.\citepls 

\subsection{Zielsetung}
Das Ziel der Arbeit soll es sein diesen Ansatz näher zu untersuchen. Im Umfeld einer SAP Lösung für die Personalwirtschaft soll im Rahmen eines Migrationsprojektes eine Applikation auf eine Cloud-Infrastruktur umgezogen werden. In diesem Kontext bietet es sich an, das Potenzial einer \ac{EDA} näher zu untersuchen. Im Verlauf dieser Arbeit soll daher ein Prototyp im SAP Kontext erstellt werden, der eine \ac{EDA} implementiert. So soll exemplarisch geprüft werden, wie der Prozess bei einem solchen Vorhaben aussehen könnte und daraus folgend ein allgemeines Urteil über den Architekturansatz gefällt werden. Der Prototyp dient somit als Grundlage für die Beantwortung der Frage, ob es sich lohnt, die \ac{EDA} in einem SAP Kontext einzusetzen.

\subsection{Betrieblicher Kontext}
Die SAP SE ist ein deutsches Softwareunternehmen, das seit 1972 Unternehmenssoftware entwickelt. Heute beschäftigt es rund 105000 Mitarbeiter und hat Standorte weltweit. Die \ac{ERP} Systeme der Firma haben in der Geschäftswelt entscheidenden, branchenübergreifenden Einfluss. SAP bietet hierbei Möglichkeit, durch umfassende Funktionen und eine einheitliche, integrierte Datenbasis, Geschäftsprozesse zu überblicken, dieses digital abzuwickeln und zu automatisieren. \footcite[Vgl.][]{sapse_was} \\
Das HCM ist die Personallösung des SAP ERP und kommt bis heute in einer großen Anzahl Firmen zum Einsatz. Besonders relevant sind hierbei sogenannte Self-Services über die Mitarbeiter Daten pflegen können und HR-relevante Prozesse anstoßen können. Dementsprechend stehen diesen Anwendungen besonders hohe Ansprüche an Leistungsfähigkeit und Verfügbarkeit entgegen.

\subsection{Abgrenzung}

\subsection{Vorgehensweise und Aufbau der Arbeit}
Um die genannte Frage zu beantworten, ist es als ersten Schritt der Arbeit notwendig, den aktuellen Forschungsstand näher zu untersuchen und die Grundlagen sowohl der \ac{EDA} als auch den technischen Kontext einer RESTful API zu erarbeiten. Dafür sollen in einer strukturierten Literaturrecherche die wissenschaftlichen Grundlagen erarbeitet werden. Die Struktur dieses theoretischen Teils ordnet sich nach Thema. Darauf folgend soll die Entwicklung des oben erwähnten Prototypen beschrieben werden. Hierbei soll zuerst ein abstrahierender Überblick über die Implementierungsschritte gegeben werden und darauf folgend ein Schluss über das Ergebnis gezogen werden. Auf Basis der daraus gewonnen Erkenntnisse und in Verknüpfung mit den vorher dargelegten theoretischen Grundlagen sollen dann weiterführende Schlüsse zur Beantwortung der Fragestellung verdichtet werden.