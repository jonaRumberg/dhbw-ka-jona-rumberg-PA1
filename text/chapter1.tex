\section{Einleitung}

\subsection{Motivation und Problemstellung}
Die wohl umfassendste Entwicklung in der globalen \ac{IT}-Branche ist heutzutage wohl die in Richtung Cloud. Um Geschäftsprozesse in diesem Umfeld umfassend abzubilden ist es immer wieder nötig verschiedene Dienste miteinander zu integrieren und so ist die Entwicklung sogenannter APIs zu einem Kernbestandteil der modernen Entwicklungstätigkeit geworden.
\\Ein weit verbreitetes Design Pattern ist dabei die sogenannte \ac{REST} \ac{API}. In der Migration von ehemals sequenziell ablaufenden Prozessen kann es jedoch zu Hürden kommen, da eine \ac{REST} \ac{API} von Grund auf stateless angelegt sein sollte. Hier kommt Event basiertes Design ins Spiel, das dieses Problem lösen könnte.



\subsection{Zielsetung}
Das Ziel dieser Arbeit ist die Untersuchung vom Zusammenspiel des eventbasierten Ansatzes und dem \ac{REST}ful design pattern, wobei speziellen Wert auf die Betrachtung der Robustheit und Fehlerresillienz des entstehenden Systems gelegt werden soll. Zudem soll ein solches System im Kontext einer HR-Anwendung beispielhaft umgesetzt werden.


\subsection{Betrieblicher Kontext}
Die SAP SE ist ein deutsches Softwareunternehmen, das seit 1972 Unternehmenssoftware entwickelt. Heute beschäftigt es rund 105000 Mitarbeiter und hat Standorte weltweit. Die \ac{ERP} Systeme der Firma haben in der Geschäftswelt entscheidenden, branchenübergreifenden Einfluss. SAP bietet hierbei Möglichkeit, durch umfassende Funktionen und eine einheitliche, integrierte Datenbasis, Geschäftsprozesse zu überblicken, dieses digital abzuwickeln und zu automatisieren. \footcite[Vgl.][]{sapse_was}

\subsection{Abgrenzung}
\subsection{Vorgehensweise und Aufbau der Arbeit}
    %Zwei drei Worte zu Forschungsmethodik
    