\section{Einleitung}
\subsection{Motivation und Problemstellung}
Eine der größten Schwierigkeiten bei der Entwicklung von Softwareprojekten im größeren Maßstab ist es, die Komplexität der Software zu beherrschen. Schnell passiert es, dass über die Verbindung von vielen verschiedenen Softwarekomponenten mit verschiedenen Schnittstellen die Komplexität der Software rapide ansteigt, was die Entwicklung und den Betrieb erschweren. \cite[Vgl.][S. 5]{pressman2005software} Einen Ansatz zur Beherrschung dieser Komplexität bietet die \ac{EDA}. Sie verspricht, durch den Fokus auf Ereignisse beim Systementwurf eine Reihe von Vorteilen. In der Prozessmodellierung lassen sich Geschäftsvorfälle einfacher modellieren, in der Implementierung wird von Beginn an eine modulare Struktur geschaffen, die Ausfallsicherheit, Integrationsmöglichkeiten und eine bessere Lesbarkeit des Programmcodes bietet. \cite[Vgl. ][S. 8f]{EDA2010} \\
Den betrieblichen Kontext dieser Überlegungen stellt die SAP SE. Die SAP SE ist ein deutsches Softwareunternehmen, das seit 1972 Unternehmenssoftware entwickelt. Die \ac{ERP} Systeme der Firma haben in der Geschäftswelt entscheidenden, branchenübergreifenden Einfluss und bieten die Möglichkeit, Geschäftsprozesse zu überblicken, dieses digital abzuwickeln und zu automatisieren. \cite[Vgl.][]{sapse_was} 
Das HCM ist die Personallösung des SAP ERP und kommt bis heute in einer großen Anzahl von Firmen zum Einsatz. Dementsprechend stehen ihren Anwendungen besonders hohe Ansprüche an Leistungsfähigkeit und Verfügbarkeit entgegen, die sie mit einer immensen Größe und Anzahl von Komponenten vereinen müssen.

\subsection{Zielsetung}
Das Ziel der Arbeit soll es sein diesen Ansatz der \acl{EDA} näher zu untersuchen. Im Umfeld von HCM soll im Rahmen eines Migrationsprojektes eine Applikation von Fiori Freestyle auf Fiori Elements unter Verwendung des SAP eigenen \ac{REST} Frameworks, dem \ac{RAP} umgezogen werden. In diesem Kontext bietet es sich an, das Potenzial einer \ac{EDA} näher zu untersuchen. Im Verlauf dieser Arbeit soll daher ein Prototyp im SAP Kontext erstellt werden, der eine \ac{EDA} implementiert. So soll exemplarisch geprüft werden, wie der Prozess bei einem solchen Vorhaben aussehen könnte und daraus folgend ein allgemeines Urteil über den Architekturansatz gefällt werden. Der Prototyp dient somit als Grundlage für die Beantwortung der verallgemeinerbaren Überlegung, ob es sich lohnt, die \ac{EDA} in komplexen Softwareumfeldern einzusetzen.

\subsection{Abgrenzung}
In der Arbeit soll es darum gehen, einen Protoypen zu bauen und zu erforschen, wie sich die Anwendung von \ac{EDA} in komplexen Systemen auswirkt. Es soll keine theoretische Grundlagenforschung zum Architekturkonzept betrieben werden oder eine umfassende Analyse der Vor- und Nachteile aus theoretischer Perspektive durchgeführt werden. Die Arbeit soll sich auf die praktische Anwendung des Konzepts beschränken und die gewonnenen Erkenntnisse in einen betriebswirtschaftlichen Kontext einordnen. Der Ausdruck komplexes Softwareumfeld bezieht sich in diesem Fall auf eine Architektur, die sich auf Prinzipien von \ac{REST} beziehen. Dabei sollen grundlegende Konzepte von \ac{REST} erarbeitet werden, der Fokus der Arbeit bleibt aber auf der Untersuchung der \ac{EDA}.

\subsection{Vorgehensweise und Aufbau der Arbeit}
Um die Frage, ob es sich lohnt, \ac{EDA} in komplexen Softwareumfeldern einzusetzen, zu beantworten, ist es als ersten Schritt der Arbeit notwendig, den aktuellen Forschungsstand näher zu untersuchen und die Grundlagen sowohl der \ac{EDA} als auch den technischen Kontext einer RESTful API zu erarbeiten. Dafür sollen in einer strukturierten Literaturrecherche die wissenschaftlichen Grundlagen erarbeitet werden. Darauffolgend soll die Entwicklung des oben erwähnten Prototypen beschrieben werden. Hierbei soll zuerst ein abstrahierender Überblick über den Erkenntnisgewinn im Rahmen der Implementierung gegeben werden und darauffolgend ein Schluss über das Ergebnis gezogen werden. Auf Basis der daraus gewonnen Erkenntnisse und in Verknüpfung mit den theoretischen Grundlagen soll die Forschungsfrage beantwortet werden. 