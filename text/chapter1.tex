\section{Einleitung}
\subsection{Motivation und Problemstellung}
Eine der größten Schwierigkeiten bei der Entwicklung von Software ist der Umgang mit Komplexität. Gerade in großen Systemen mit vielen Komponenten und Schnittstellen wächst die Komplexität der Software oft rapide an und führt zu gesteigertem Entwicklungsaufwand, einem höheren Ausfallrisiko und somit direkt zu Mehrkosten. Deshalb gilt es diese Tendenz bereits in der Konzeptionierung des Systems zu antizipieren und auf Ebene der Softwarearchitektur anzugehen. Ein Architekturansatz, der die wachsende Komplexität von Software bei wachsender Größe adressiert, ist die EDA. Es handelt sich hierbei um ein Architekturmodell, dass inhärent auf entkoppelte Komponenten, asynchrone Prozesse und technologische Unabhängigkeit setzt. Das Potenzial von \ac{EDA} soll im Anwendungskontext näher untersucht werden. \\ Der Anwendungskontext ist das HCM, eine Personallösung der SAP. Die SAP SE ist ein deutsches Softwareunternehmen, das seit 1972 Unternehmenssoftware entwickelt. Die \ac{ERP} Systeme der Firma haben in der Geschäftswelt entscheidenden, branchenübergreifenden Einfluss und bieten die Möglichkeit, Geschäftsprozesse zu überblicken, dieses digital abzuwickeln und zu automatisieren. \cite[Vgl.][]{sapse_was} Dem \ac{HCM} als Personallösung, welche in einer großen Anzahl von Unternehmen im Einsatz ist, stehen hierbei besonders hohe Anforderungen an Stabilität und Effizienz der Software entgegen. Die Fragestellung, die sich in diesem Kontext für die Arbeit ergibt, ist, ob es sich lohnt, die \ac{EDA} in komplexen Softwareumfeldern, wie eben dem \ac{HCM}, einzusetzen.
Den Rahmen der Betrachtung bildet ein Migrationsprojekt der Software auf das SAP eigene REST Framework RAP. Konkret ist die asynchrone Kommunikation zwischen verschiedenen Komponenten im Rahmen dieser Migration von Interesse. Sie würde eine Beschränkung von RAP umgehen, die besagt, dass in einer Speichersequenz einer Komponente keine Speichersequenz einer anderen Komponente synchron aufgerufen werden darf.

\subsection{Zielsetzung}
Das Ziel der Arbeit soll es sein, das Potenzial der \acl{EDA} näher zu untersuchen. Im Verlauf der Arbeit soll ein Prototyp im SAP Kontext erstellt werden, der eine \ac{EDA} implementiert. Zum einen soll dadurch ein explorativer Erkenntnisgewinn erzielt werden und zum anderen exemplarisch geprüft werden, wie der Prozess bei einem solchen Vorhaben aussehen könnte. Daraus folgend soll ein allgemeines Urteil über den Architekturansatz gefällt werden. So kann auch geklärt werden, ob der Einsatz einer \ac{EDA} eine valide Möglichkeit ist, die genannte Beschränkung von RAP umzusetzen. Der Prototyp dient aber auch als Grundlage für die Beantwortung der verallgemeinerbaren Überlegung, ob es sich lohnt, die \ac{EDA} in umfangreichen, also komplexen Softwareumfeldern einzusetzen.

\subsection{Abgrenzung}
In der Arbeit soll es darum gehen, diesen Prototypen zu implementieren und zu erforschen, wie sich die Anwendung von \ac{EDA} in komplexen Systemen auswirkt. Es soll keine theoretische Grundlagenforschung zum Architekturkonzept betrieben werden oder eine umfassende Analyse der Vor- und Nachteile aus theoretischer Perspektive durchgeführt werden. Die Arbeit soll sich auf die praktische Anwendung des Konzepts beschränken und die gewonnenen Erkenntnisse in einen betriebswirtschaftlichen Kontext einordnen. Der Ausdruck komplexes Softwareumfeld bezieht sich in diesem Fall auf eine Architektur, die in erster Linie einen gewissen Umfang, also Anzahl von Komponenten und Schnittstellen besitzt. In Konsequenz der praktischen Betrachtung eines bestehenden Systems liegt aber vor allem dieses konkrete System der Betrachtung zugrunde. Da dieses System auf Basis der Prinzipien von REST konzeptioniert ist, sollen auch grundlegende Konzepte von \ac{REST} erarbeitet werden, der Fokus der Arbeit bleibt aber auf der Untersuchung der \ac{EDA}.

\subsection{Vorgehensweise und Aufbau der Arbeit}
Um die Frage, ob es sich lohnt, \ac{EDA} in komplexen Softwareumfeldern einzusetzen, zu beantworten, ist es als ersten Schritt der Arbeit notwendig, den aktuellen Forschungsstand zum Thema darzustellen und die Grundlagen sowohl der \ac{EDA} als auch den technischen Kontext einer RESTful API zu erarbeiten. Dafür sollen in einer Literaturrecherche die wissenschaftlichen Grundlagen erarbeitet werden. Darauffolgend soll die Entwicklung des oben erwähnten Prototypen beschrieben werden. Hierbei soll zuerst ein abstrahierender Überblick über den Erkenntnisgewinn im Rahmen der Implementierung gegeben werden und darauffolgend ein Schluss über das Ergebnis gezogen werden. Auf Basis der daraus gewonnen Erkenntnisse und in Verknüpfung mit den theoretischen Grundlagen soll die Forschungsfrage beantwortet werden. 